\documentclass{article}

\usepackage{listings}
\usepackage{xcolor}
\usepackage[margin=1in]{geometry} % <-- ADD THIS LINE to set 1-inch margins
\usepackage{graphicx}

\definecolor{codegreen}{rgb}{0,0.6,0}
\definecolor{codegray}{rgb}{0.5,0.5,0.5}
\definecolor{codepurple}{rgb}{0.58,0,0.82}
\definecolor{backcolour}{rgb}{0.95,0.95,0.92}

\lstdefinestyle{mystyle}{
    backgroundcolor=\color{backcolour},   
    commentstyle=\color{codegreen},
    keywordstyle=\color{magenta},
    numberstyle=\tiny\color{codegray},
    stringstyle=\color{codepurple},
    basicstyle=\ttfamily\footnotesize,
    breakatwhitespace=false,         
    breaklines=true,                 
    captionpos=b,                    
    keepspaces=true,                 
    numbers=left,                    
    numbersep=5pt,                  
    showspaces=false,                
    showstringspaces=false,
    showtabs=false,                  
    tabsize=2,
    columns=fullflexible,   % <-- ADD THIS to help with alignment
    xleftmargin=0pt,        % <-- SET to 0pt for left flush
    xrightmargin=0pt        % <-- SET to 0pt
}

\lstset{style=mystyle}

\title{Relational Model for a construction company}

\author{Ashwin Ajoy Dharmavaram - 1RV22AI011}

\date{}

\begin{document}

\maketitle

Some information required to make a relational model for a construction company is:

\begin{enumerate}
    \item Can a single client take part request multiple projects at a time?
    \item How is the relationship between employees, and other employees? How are managers assigned for projects?
    \item How is payment for a contract decided? Is it by time, decided before hand, etc. 
    \item Are the suppliers always in the same region? Does the company need drivers/delivery people for transporting materials?
    \item What happens when multiple projects are related closely (same client, same location, etc) ?
\end{enumerate}

There are a lot more questions that could be asked, the above are some that relate to entities and relationships only. Depending on the questions that can be asked, the cardinality of relationships between entities can be decided. 

The below ER diagram shows what it will look like. Apart from the Supplier and Material Entities, any general contract based firms would fit the below basic ER Diagram. 
\begin{figure}[htpb]
    \centerline{\includegraphics[scale=0.2]{ERD.png}}
    \caption{ER Diagram for the Construction Company}
\end{figure}

\end{document}
